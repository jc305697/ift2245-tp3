\documentclass{article}

\usepackage[utf8]{inputenc}

\title{Travail pratique \#3 - IFT-2245}
\author{Maude Sabourin et Jérémy Coulombe}

\begin{document}

\maketitle

\section{Expérience}
Durant ce travail qui semblait si simple, nous avons été confronté à plusieurs surprises innatendues. Par exemple, nous nous sommes rendus compte que nous ne gérions pas le cas d'une écriture sur une page qui est en readonly. Par la suite, nous avons essayer d'implanter le copy on write pour nous rendre compte que notre algorithme pour trouver le frame qui a été utilisé le moins récemment ne retournait que le numéro du frame donc nous avons dû modifier celui-ci afin de gérer ce cas pour se rendre compte que ça nous donnait de très mauvais résultat au niveau du tlb miss rate.

\section {Difficultés}
Unsigned Integer : Comparaison, problématique

Commandes de style fseek, fwrite, fread, fputs

Bien été
Memset, commence à être habitué
Strncpy

Algorithme
Au début : implémentation liste 
Un peu compliqué _+ coûteux de toujours déplacer tous les éléments

Ensuite implémentation binaire, shift
Comment gérer l’ajout : utilisation du 100000000 pour 64 bits 
Même algo LRU et PT

Bons rendements

Implémentation copy on write et gestion readonly

\section{Résultats}


\section{Algo et choix implantés}
Backup juste si dirty
Copie on write baisse les perfos (plus réaliste)
\section{Améliorations possibles}


\section{Fonctionnement du code}

\end{document}
